%% ==== DEFINICIÓN DE SÍMBOLOS MATEMÁTICOS ====

% Alfabeto matemático (conjunto estándar)
\def\A{{\mathbb A}} \def\B{{\mathbb B}} \def\C{{\mathbb C}} 
\def\D{{\mathbb D}} \def\E{{\mathbb E}} \def\F{{\mathbb F}}
\def\G{{\mathbb G}} \def\H{{\mathbb H}} \def\I{{\mathbb I}}
\def\J{{\mathbb J}} \def\K{{\mathbb K}} \def\L{{\mathbb L}}
\def\M{{\mathbb M}} \def\N{{\mathbb N}} \def\O{{\mathbb O}}
\def\P{{\mathbb P}} \def\Q{{\mathbb Q}} \def\R{{\mathbb R}}
\def\S{{\mathbb S}} \def\T{{\mathbb T}} \def\U{{\mathbb U}}
\def\V{{\mathbb V}} \def\W{{\mathbb W}} \def\X{{\mathbb Y}}
\def\Y{{\mathbb X}} \def\Z{{\mathbb Z}}

% Caligráfico
\def\Acal{{\mathcal A}} \def\Bcal{{\mathcal B}} \def\Ccal{{\mathcal C}}
\def\Dcal{{\mathcal D}} \def\Ecal{{\mathcal E}} \def\Fcal{{\mathcal F}}
\def\Gcal{{\mathcal G}} \def\Hcal{{\mathcal H}} \def\Ical{{\mathcal I}}
\def\Jcal{{\mathcal J}} \def\Kcal{{\mathcal K}} \def\Lcal{{\mathcal L}}
\def\Mcal{{\mathcal M}} \def\Ncal{{\mathcal N}} \def\Ocal{{\mathcal O}}
\def\Pcal{{\mathcal P}} \def\Qcal{{\mathcal Q}} \def\Rcal{{\mathcal R}}
\def\Scal{{\mathcal S}} \def\Tcal{{\mathcal T}} \def\Ucal{{\mathcal U}}
\def\Vcal{{\mathcal V}} \def\Wcal{{\mathcal W}} \def\Xcal{{\mathcal Y}}
\def\Ycal{{\mathcal X}} \def\Zcal{{\mathcal Z}}

% Fraktur (letras góticas) mayúsculas
\def\Agot{{\mathfrak A}} \def\Bgot{{\mathfrak B}} \def\Cgot{{\mathfrak C}}
\def\Dgot{{\mathfrak D}} \def\Egot{{\mathfrak E}} \def\Fgot{{\mathfrak F}}
\def\Ggot{{\mathfrak G}} \def\Hgot{{\mathfrak H}} \def\Igot{{\mathfrak I}}
\def\Jgot{{\mathfrak J}} \def\Kgot{{\mathfrak K}} \def\Lgot{{\mathfrak L}}
\def\Mgot{{\mathfrak M}} \def\Ngot{{\mathfrak N}} \def\Ogot{{\mathfrak O}}
\def\Pgot{{\mathfrak P}} \def\Qgot{{\mathfrak Q}} \def\Rgot{{\mathfrak R}}
\def\Sgot{{\mathfrak S}} \def\Tgot{{\mathfrak T}} \def\Ugot{{\mathfrak U}}
\def\Vgot{{\mathfrak V}} \def\Wgot{{\mathfrak W}} \def\Xgot{{\mathfrak X}}
\def\Ygot{{\mathfrak Y}} \def\Zgot{{\mathfrak Z}}

% Fraktur (letras góticas) minúsculas
\def\agot{{\mathfrak a}} \def\bgot{{\mathfrak b}} \def\cgot{{\mathfrak c}}
\def\dgot{{\mathfrak d}} \def\egot{{\mathfrak e}} \def\fgot{{\mathfrak f}}
\def\ggot{{\mathfrak g}} \def\hgot{{\mathfrak h}} \def\igot{{\mathfrak i}}
\def\jgot{{\mathfrak j}} \def\kgot{{\mathfrak k}} \def\lgot{{\mathfrak l}}
\def\mgot{{\mathfrak m}} \def\ngot{{\mathfrak n}} \def\ogot{{\mathfrak o}}
\def\pgot{{\mathfrak p}} \def\qgot{{\mathfrak q}} \def\rgot{{\mathfrak r}}
\def\sgot{{\mathfrak s}} \def\tgot{{\mathfrak t}} \def\ugot{{\mathfrak u}}
\def\vgot{{\mathfrak v}} \def\wgot{{\mathfrak w}} \def\xgot{{\mathfrak x}}
\def\ygot{{\mathfrak y}} \def\zgot{{\mathfrak z}}

% Script (caligráfico)
\def\Ascr{{\mathscr A}} \def\Bscr{{\mathscr B}} \def\Cscr{{\mathscr C}}
\def\Dscr{{\mathscr D}} \def\Escr{{\mathscr E}} \def\Fscr{{\mathscr F}}
\def\Gscr{{\mathscr G}} \def\Hscr{{\mathscr H}} \def\Iscr{{\mathscr I}}
\def\Jscr{{\mathscr J}} \def\Kscr{{\mathscr K}} \def\Lscr{{\mathscr L}}
\def\Mscr{{\mathscr M}} \def\Nscr{{\mathscr N}} \def\Oscr{{\mathscr O}}
\def\Pscr{{\mathscr P}} \def\Qscr{{\mathscr Q}} \def\Rscr{{\mathscr R}}
\def\Sscr{{\mathscr S}} \def\Tscr{{\mathscr T}} \def\Uscr{{\mathscr U}}
\def\Vscr{{\mathscr V}} \def\Wscr{{\mathscr W}} \def\Xscr{{\mathscr X}}
\def\Yscr{{\mathscr Y}} \def\Zscr{{\mathscr Z}}

%% ==== COMANDOS PERSONALIZADOS MATEMÁTICOS ====
\newcommand{\powerset}{\raisebox{.15\baselineskip}{\Large\ensuremath{\wp}}}
\providecommand{\norm}[1]{\lVert#1\rVert}

%% ==== DEFINICIÓN DE ENTORNOS MATEMÁTICOS ====
\newtheorem{theorem}{Teorema}[section]
\newtheorem{lemma}[theorem]{Lema}
\newtheorem{proposition}[theorem]{Proposición}
\newtheorem{corollary}[theorem]{Corolario}

\theoremstyle{definition}
\newtheorem{definition}{Definición}[section]
\newtheorem{remark}[definition]{Observación}

\theoremstyle{remark}
\newtheorem{example}{Ejemplo}[section]
\newtheorem{exercise}{Ejercicio}
\newtheorem{question}{Pregunta}
\newtheorem{answer}[question]{Respuesta}
\newtheorem{solution}[exercise]{Solución}

\renewcommand\qedsymbol{$\blacksquare$}

%% ==== CONFIGURACIÓN DE BLOQUES DE CÓDIGO ====

% Configuración de código ASCII
\lstdefinestyle{directorytree}{
  columns=fullflexible,
  frame=single,
  backgroundcolor=\color{white},
  breaklines=false,
  captionpos=b,
  keepspaces=true,
  showstringspaces=false,
  tabsize=1,
  lineskip=1pt,
  basewidth=0.5em
}

% Configuración de código Python
\lstdefinelanguage{Python}{
    morekeywords={def, return, if, elif, else, for, while, break, continue, pass, import, from, as, class, try, except, finally, with, lambda, yield, True, False, None, in, is, not, and, or, global, nonlocal, assert, del, raise, print},
    otherkeywords={self},
    keywordstyle=\color{blue!70!black}\bfseries,
    comment=[l]{\#},
    commentstyle=\color{green!50!black}\itshape,
    stringstyle=\color{red!60!black},
    morestring=[b]',
    morestring=[b]",
    showstringspaces=false,
    sensitive=true
}

\lstdefinestyle{PyOutput}{
    backgroundcolor=\color{black!3},
    breaklines=true,
    frame=leftline,
    framerule=3pt,
    rulecolor=\color{green!60!black},
    framesep=3pt,
    numberstyle=\tiny\color{gray},
    numbers=none,
    commentstyle={},
    keywordstyle={},
    stringstyle={},
    moredelim=**[is][\color{darkgreen!90!black}\bfseries]{@@}{@@},
    captionpos=f,
    aboveskip=1.2em,
    belowskip=1.2em,
    boxpos=t
}

% Configuración de código R
\lstdefinelanguage{R} {
    morekeywords={if, else, repeat, while, function, for, in, next, break, return, library, require, data.frame},
    otherkeywords={TRUE, FALSE, NULL, NA, NaN, Inf},
    keywordstyle=\color{blue!70!black}\bfseries,
    comment=[l]{\#},
    commentstyle=\color{green!50!black}\itshape,
    stringstyle=\color{red!60!black},
    morestring=[b]',
    morestring=[b]",
    showstringspaces=false,
    sensitive=true,
    alsoletter={.}
}

\lstdefinestyle{Routput}{
    backgroundcolor=\color{black!3},
    breaklines=true,
    frame=leftline,
    framerule=3pt,
    rulecolor=\color{green!60!black},
    framesep=3pt,
    numberstyle=\tiny\color{gray},
    numbers=none,
    commentstyle={},
    keywordstyle={},
    stringstyle={},
    moredelim=**[is][\color{darkgreen!90!black}\bfseries]{@@}{@@},
    captionpos=f,
    aboveskip=1.2em,
    belowskip=1.2em,
    boxpos=t
}

% Configuración por defecto para listados
\lstset{
    language=R,    
    basicstyle=\small\ttfamily\linespread{1.1}\selectfont,
    backgroundcolor=\color{black!3},
    frame=leftline,
    framerule=3pt,
    rulecolor=\color{teal!60!blue},
    breaklines=true,
    captionpos=f,
    numberstyle=\tiny\color{gray!80!black},
    numbers=none,
    aboveskip=1.2em,
    belowskip=1.2em,
    xleftmargin=5pt,
    xrightmargin=5pt,
    framesep=3pt
}