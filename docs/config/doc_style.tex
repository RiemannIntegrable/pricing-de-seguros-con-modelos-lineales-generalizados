%% ==== CONFIGURACIÓN DE PÁGINA ====
\usepackage[papersize={216mm, 279mm},tmargin=20mm,bmargin=20mm,lmargin=20mm,rmargin=20mm]{geometry}
\setlength{\parindent}{0mm}

%% ==== CONFIGURACIÓN DE FUENTES ====
\renewcommand{\familydefault}{\sfdefault}

%% ==== ENCABEZADO Y PIE DE PÁGINA ====
\pagestyle{fancy}
\fancyhf{}
\lhead{Universidad Nacional De Colombia}
\rhead{Facultad De Ciencias}
\cfoot{\thepage\ de \pageref{LastPage}}
\lfoot{Departamento De Matemáticas}
\rfoot{Modelos Lineales Generalizados para }
\renewcommand{\headrulewidth}{0.08pt}
\renewcommand{\footrulewidth}{0.08pt}

%% ==== ESTILO DE PRIMERA PÁGINA ====
\fancypagestyle{firstpage}{
  \renewcommand{\headrulewidth}{0cm}
  \cfoot{\thepage\ de \pageref{LastPage}}
  \lfoot{Pricing de seguros con GLMs}
  \rfoot{\today}
  \lhead{}
  \rhead{}
  \chead{
      \begin{tabular}{|ccc|r|}
           \hline
           \multirow{13}{*}{\includegraphics[scale=0.13]{../images/Escudo_de_la_Universidad_Nacional_de_Colombia_(2016).png}} &  &  &  \\
           & {\LARGE Universidad Nacional de Colombia} &  & \textbf{Estudiantes:} \\
           &  &  & Jose Miguel Acuña Hernandez \\
           &  {\Large Facultad de ciencias} &  & jacunah@unal.edu.co \\
           &  &  &  \\
           &  {\Large Departamento de matemáticas} &  & Guillermo Murillo Tirado \\
           &  &  & gmurillot@unal.edu.co \\
           & {\Large Modelos lineales generalizados} &  &  \\
           & {\Large para seguros} &  & \textbf{Docente:} \\
           &  &  & Luz Mery Gonzalez G. \\
           & {\Large Pricing de seguros utilizando}  &  & lgonzalezg@unal.edu.co \\
           & {\Large modelos lineales generalizados} &  &  \\
           &  &  &  \\
           \hline
      \end{tabular}\\
  }
}

%% ==== ESTILO DE CONTENIDO ====
\newenvironment{cuadrocontenido}{
  \begin{center}
  \begin{tabular}{|p{0.97\textwidth}|}
    \hline
    \multicolumn{1}{|c|}{
        \rule{0pt}{2.5ex}
        \textbf{\large Contenido}
        \rule{0pt}{2.5ex}
    } \\
    \hline
    \begin{minipage}{0.96\textwidth}
        \centering
        \vspace{0.5em}
}{%
        \vspace{0.5em}
    \end{minipage} \\
    \hline
  \end{tabular}
  \end{center}
}

%% ==== CONFIGURACIÓN DEL CUADRO DE CONTENIDO ====
\renewcommand{\contentsname}{}

%% ==== ESPACIADO Y FORMATO INICIAL ====
% Espaciado inicial en el documento
\newcommand{\espacioinicial}{\vspace*{10\baselineskip}}