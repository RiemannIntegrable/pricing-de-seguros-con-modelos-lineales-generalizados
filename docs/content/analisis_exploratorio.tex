\section{Análisis Exploratorio de Datos}

\subsection{Descripción del Dataset}

El dataset analizado contiene información de 5,000 pólizas de seguros de automóviles con 18 variables, de las cuales se identificaron tres variables numéricas (Modelo, Edad del asegurado, Valor comercial), once variables categóricas incluyendo la variable objetivo, y dos variables de fechas para el cálculo de exposición temporal.

\begin{table}[H]
\centering
\caption{Estructura del Dataset}
\begin{tabular}{|l|c|l|}
\hline
\textbf{Tipo de Variable} & \textbf{Cantidad} & \textbf{Variables Clave} \\
\hline
Numéricas & 3 & Modelo, Edad, Vr\_Comercial \\
Categóricas & 11 & SERVICIO, Sexo\_Aseg, TIPO\_VEHICULO, MARCA \\
Fechas & 2 & Desde, Hasta (período de vigencia) \\
Variable Objetivo & 1 & Pago (Si/No - ocurrencia de siniestro) \\
\hline
\textbf{Total} & \textbf{18} & \textbf{Registros: 5,000} \\
\hline
\end{tabular}
\end{table}

Dado que no tenemos un columna que indique el tipo de poliza que se tiene para cada registro se asumirá que todas las polizas en la base de datos son contra todo riezgo. Por otro lado, para facilidad y viabilidad del modelo las varaibles númericas como Modelo y Edad se van a agrupar. Las varaibles como Modelo y Edad serán agrupadas, la única variable que será siguiendo continua es la Variable de Vr\_Comercial. Las variable de tipo fecha serán usadas para calcular una variable que calcule la exposición en días. 
\subsection{Análisis de Calidad de Datos}

La evaluación de la calidad de los datos constituye un paso fundamental antes de la modelación actuarial, ya que permite identificar problemas que pueden comprometer la validez del modelo GLM. El análisis revela diversos aspectos críticos que requieren tratamiento previo a la construcción del modelo.

\subsubsection{Valores Faltantes}

\begin{table}[H]
\centering
\caption{Análisis Detallado de Valores Faltantes}
\begin{tabular}{|l|c|c|l|}
\hline
\textbf{Variable} & \textbf{Valores Faltantes} & \textbf{Porcentaje} & \textbf{Impacto Actuarial} \\
\hline
Amparo & 4,320 & 86.4\% & Solo para pólizas sin siniestro \\
Amp & 4,320 & 86.4\% & Solo para pólizas sin siniestro \\
SumaDePagos & 4,320 & 86.4\% & Severidad - No crítico para frecuencia \\
Sexo\_Aseg & 625 & 12.5\% & Variable demográfica - Requiere imputación \\
Otras variables explicativas & 0 & 0.0\% & Excelente calidad para modelación \\
\hline
\end{tabular}
\end{table}

Los valores faltantes en variables de severidad (Amparo, SumaDePagos) corresponden exclusivamente a pólizas sin siniestros, lo cual es esperado y no representa un problema para el modelo de frecuencia. Sin embargo, el 12.5\% de valores faltantes en la variable sexo del asegurado requiere tratamiento mediante técnicas de imputación o eliminación, considerando su importancia como factor de riesgo demográfico.

\subsubsection{Problemas Críticos Identificados}

Se identificaron 795 registros (15.9\% del dataset) con valores de edad igual a 0, lo cual constituye un problema crítico que compromete la modelación.
Adicionalmente, se observaron 522 registros (10.44\% del dataset) con edades superiores a 80 años, incluyendo valores extremos hasta 999 años. Al hacer la agrupación de esta categoría estos datos serán tratado como una categoría aparte.
\newline


Similarmente, se identificaron 308 registros (6.16\% del dataset) con valor comercial igual a 0. Estos registros comprometen el cálculo de primas basadas en valor del vehículo
\newline

EN cuanto a las variable categóricas, hay en la mayoria de ellas un exeso de grupos. Por consiguiente, el tratatamiento de los datos que resultaron relevantes como el Color será reagrupar los datos según su frecuencia; se establecío el número máximo de grupos como 4, esto con el objetico de faciliar el calculo de las tarifas. 

\subsection{Variables Objetivo: Frecuencia de Siniestros y Suma de Pagos}

La distribución de la variable objetivo presenta características fundamentales para el diseño del modelo GLM:

\begin{table}[H]
\centering
\caption{Distribución de la Variable Objetivo}
\begin{tabular}{|l|c|c|}
\hline
\textbf{Ocurrencia de Siniestro} & \textbf{Frecuencia} & \textbf{Porcentaje} \\
\hline
No & 4,485 & 89.7\% \\
Sí & 515 & 10.3\% \\
\hline
\textbf{Total} & \textbf{5,000} & \textbf{100.0\%} \\
\hline
\end{tabular}
\end{table}

La tasa de siniestralidad global del 10.3\% se encuentra dentro de los rangos típicos del mercado asegurador colombiano, proporcionando una base sólida para la estimación de parámetros en el modelo GLM binomial. Esta frecuencia garantiza suficiente variabilidad para la estimación robusta de coeficientes.

\subsection{Variables Explicativas Críticas}

\subsubsection{Servicio del Vehículo}

El análisis por tipo de servicio revela diferencias estadísticamente significativas en la siniestralidad, constituyendo el factor de segmentación más relevante identificado:

\begin{figure}[H]
\centering
\includegraphics[width=0.9\textwidth]{../images/analisis_servicio_vehiculo.png}
\caption{Análisis de Siniestralidad por Servicio del Vehículo}
\end{figure}

\begin{table}[H]
\centering
\caption{Siniestralidad por Tipo de Servicio}
\begin{tabular}{|l|c|c|c|}
\hline
\textbf{Servicio} & \textbf{Total Pólizas} & \textbf{Siniestros} & \textbf{Tasa (\%)} \\
\hline
Transporte Personal & 7 & 1 & 14.286 \\
Oficial & 33 & 4 & 12.121 \\
Particular & 4,133 & 432 & 10.440 \\
Público & 827 & 78 & 9.177 \\
\hline
\end{tabular}
\end{table}

Las diferencias observadas reflejan patrones de exposición al riesgo diferenciados: vehículos de transporte personal y oficial presentan mayor intensidad de uso y, consecuentemente, mayor probabilidad de siniestro. Esta variable debe ser considerada obligatoria en el modelo GLM.

\subsubsection{Edad del Asegurado}

El análisis por rangos etarios evidencia la curva de riesgo típica de seguros de automóviles, con mayor siniestralidad en los extremos de edad:

\begin{figure}[H]
\centering
\includegraphics[width=0.9\textwidth]{../images/analisis_edad_siniestralidad.png}
\caption{Análisis de Siniestralidad por Edad del Asegurado}
\end{figure}

\begin{table}[H]
\centering
\caption{Siniestralidad por Rango de Edad}
\begin{tabular}{|l|c|c|c|c|}
\hline
\textbf{Rango Edad} & \textbf{Total Pólizas} & \textbf{Siniestros} & \textbf{Tasa (\%)} & \textbf{Exposición (\%)} \\
\hline
18-25 & 348 & 42 & 12.069 & 8.54 \\
26-35 & 2,185 & 218 & 9.977 & 53.62 \\
36-45 & 1,055 & 116 & 10.995 & 25.90 \\
46-55 & 224 & 29 & 12.946 & 5.50 \\
56-65 & 262 & 34 & 12.977 & 6.43 \\
\hline
\end{tabular}
\end{table}

La curva en U observada confirma el comportamiento esperado: conductores jóvenes (18-25 años) y de edad intermedia-superior (56-65 años) presentan mayor riesgo relativo. Existe un problema crítico con 795 registros (15.9\%) que presentan edad igual a 0, requiriendo limpieza previa a la modelación.

\subsubsection{Valor Comercial del Vehículo}

El valor comercial actúa como proxy de la suma asegurada y exposición económica, mostrando patrones diferenciados por rango de valor:

\begin{figure}[H]
\centering
\includegraphics[width=0.9\textwidth]{../images/analisis_valor_comercial.png}
\caption{Distribución y Siniestralidad por Valor Comercial}
\end{figure}

\begin{table}[H]
\centering
\caption{Estadísticas Descriptivas del Valor Comercial}
\begin{tabular}{|l|r|}
\hline
\textbf{Estadístico} & \textbf{Valor (COP)} \\
\hline
Media & 26,448,817 \\
Mediana & 22,400,000 \\
Desviación Estándar & 17,889,550 \\
Coeficiente de Variación & 67.6\% \\
Concentración (hasta 40M) & 78.2\% \\
\hline
\end{tabular}
\end{table}

La cartera se concentra en vehículos de gama media-baja (78.2\% con valor hasta 40 millones), lo cual es típico del mercado masivo de seguros. La alta variabilidad (CV = 67.6\%) sugiere considerar transformación logarítmica en el modelo GLM.

\subsection{Análisis de Variables Demográficas}

\subsubsection{Diferenciación por Sexo}

El análisis por sexo del asegurado muestra diferencias marginales pero estadísticamente relevantes:

\begin{figure}[H]
\centering
\includegraphics[width=0.9\textwidth]{../images/analisis_siniestralidad_sexo.png}
\caption{Siniestralidad por Sexo del Asegurado}
\end{figure}

\begin{table}[H]
\centering
\caption{Siniestralidad por Sexo}
\begin{tabular}{|l|c|c|c|}
\hline
\textbf{Sexo} & \textbf{Frecuencia} & \textbf{Siniestros} & \textbf{Tasa (\%)} \\
\hline
Femenino & 2,248 & 230 & 10.23 \\
Masculino & 2,127 & 236 & 11.10 \\
\hline
\end{tabular}
\end{table}

Aunque las diferencias son marginales (0.87 puntos porcentuales), la variables las categorías deben de verse por separado. A su vez, el 17\% de los datos no tienen sexo, por consiguiente serán tratados como otra categoría. 

\subsection{Análisis de Interacciones}

El análisis de interacciones entre edad y servicio del vehículo revela patrones complejos que justifican la inclusión de términos de interacción en el modelo GLM:

\begin{figure}[H]
\centering
\includegraphics[width=0.9\textwidth]{../images/interaccion_edad_servicio.png}
\caption{Interacción entre Edad del Asegurado y Servicio del Vehículo}
\end{figure}

La interacción muestra que el efecto de la edad varía según el tipo de servicio, siendo particularmente pronunciado en vehículos oficiales para asegurados mayores de 51 años (12.95\% de siniestralidad).
\subsection{Análisis de Correlaciones}

El análisis de correlaciones entre variables numéricas no revela multicolinealidad severa que comprometa la modelación GLM:

\begin{figure}[H]
\centering
\includegraphics[width=0.5\textwidth]{../images/output.png}
\caption{Matriz de Correlaciones entre Variables Numéricas}
\end{figure}

Las correlaciones observadas (todas menores a 0.5) se encuentran dentro de rangos aceptables para la inclusión simultánea en el modelo GLM.

\subsection{Análisis de Severidad}

Para pólizas con siniestros efectivos, el análisis de severidad por tipo de amparo proporciona información complementaria relevante:

\begin{figure}[H]
\centering
\includegraphics[width=0.9\textwidth]{../images/analisis_severidad_amparos.png}
\caption{Análisis de Severidad por Tipo de Amparo}
\end{figure}

Aunque el modelo GLM de frecuencia no utiliza directamente esta información, es relevante para el diseño integral del sistema de pricing y para validar la consistencia de los datos.