\section*{Resumen}

Este trabajo presenta el desarrollo de un sistema de tarificación actuarial para seguros de automóviles basado en Modelos Lineales Generalizados (GLM) con metodología de frecuencia-severidad. La investigación abarca desde la preparación y categorización de datos de siniestros hasta la implementación de modelos predictivos robustos, utilizando un conjunto de 6,750 observaciones de pólizas de seguros. Se implementó una metodología rigurosa de limpieza y homogeneización de la cartera mediante filtros por valor comercial, categorización óptima de variables numéricas usando el algoritmo de Freedman-Diaconis, y agrupación estratégica de variables categóricas basada en exposición. El proceso de selección de modelos empleó técnicas de forward stepwise seguidas de análisis de significancia estadística, resultando en un modelo de frecuencia quasi-Poisson con variables Modelo y Edad, y un modelo de severidad log-normal que incorpora Modelo, Color, Carrocería, Servicio y Edad del asegurado.\\

Los resultados obtenidos demuestran la efectividad del enfoque GLM para la tarificación actuarial, generando tablas de relatividades que revelan patrones de riesgo significativos: los vehículos modelo 2000-2006 presentan 29.7\% mayor frecuencia de siniestros, mientras que el servicio público incrementa la severidad en 1,508.6\% respecto al servicio particular. El sistema permite calcular 12 combinaciones de frecuencia y 768 de severidad, identificando que las primas puras más altas (hasta \$40.8 millones COP) corresponden a vehículos modelo 2007-2013 con características de alto riesgo en servicio público. La investigación culmina con la entrega de un conjunto completo de tablas actuariales y un aplicativo web interactivo en Shiny, proporcionando una herramienta práctica y técnicamente sólida para la implementación de estructuras tarifarias basadas en evidencia estadística y principios actuariales rigurosos.

